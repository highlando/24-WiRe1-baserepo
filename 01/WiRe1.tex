\documentclass[12pt]{scrartcl}
\usepackage[german]{babel}     % Neue Deutsche Rechtschreibung
\usepackage[utf8]{inputenc}    % Zeichencodierung nach UTF-8 (Unicode) - PDF\LaTeX\,\LaTeX\

\usepackage{amsmath, amssymb}  % f\"ur die Mathematik
\usepackage{xcolor}            % mehr Farben
\usepackage{graphicx}          % Bilder
\usepackage[colorlinks=true, linkcolor=teal]{hyperref}  % Links im Text und ins web

\begin{document}


\begin{tabular}{lp{3.3cm}r}
\textbf{Dr. Axel Wolf}&&\textbf{TU Ilmenau}\\ \textbf{Dr. Jan Heiland} &&\\
\textbf{Wintersemester 2024/25}&& \textbf{Institut f\"ur Mathematik}
\end{tabular}

%\vsph
\begin{center}
{\Large \textbf{ Wissenschaftliches Rechnen 1 - \"Ubungsblatt 1}}\\[1em]
{\large Besprechung am 17. Oktober 2024}
\\ \textbf{
--------------------------------------------------------------------------------------------------}
\end{center}

\tableofcontents

\section{Los geht's} Besorgen Sie sich eine \LaTeX{} Umgebung (z.B.
\emph{TexMaker})
oder melden Sie sich mittels SSO\footnote{\emph{single sign on}} und ihrem Uni
account bei \emph{overleaf} an.

\section{Wir fangen einfach an}
\begin{enumerate}
  \item Starten Sie mit einem minimalen \LaTeX template%
    \footnote{z.B. von hier
    \href{https://www.overleaf.com/read/bbdzfwnxkyyp\#1976b0}{im overleaf} oder
  hier \href{https://github.com/highlando/24-WiRe1-baserepo/blob/main/wire-1-template.tex}{bei
Github}
}
    und bereiten Sie diese
    Abgabe vor, indem sie schon mal die Kapitel\"uberschriften hinschreiben.
\item Erzeugen Sie ein Inhaltsverzeichnis und setzen Sie es vor den Text.
\end{enumerate}

\section{Es ist ganz einfach}
\textit{Schreiben Sie zu jedem \emph{Kapitel/jeder Aufgabe} einen motivierenden oder
erkl\"arenden Satz.} (bitte in kursiver Schrift)

\section{(M)ein Lebenslauf}  
\begin{enumerate}
\item 
Schreiben Sie in \LaTeX\ einen (fiktiven) tabellarisch aufgebauten Lebenslauf in zwei Spalten. 
\item
Setzen Sie eine Fu"snote mit einem Hinweis auf den Geburtsort an das Geburtsdatum.
\item
Binden Sie ein (fiktives) Foto über die Tabelle und beschriften Sie es.
\end{enumerate}

Kein Problem: siehe Tabelle \ref{tab:meinlebenslauf}.

\begin{table}[t]
  \includegraphics[width=.1\textwidth]{siegel-deutsch\_jpeg.jpg}

  \begin{tabular}{|l|r|}
    \hline
    \textbf{Name:} & Jan \\
    \hline
    \hline
    Geboren & vor ein paar Jahren\footnotemark \\
    \hline
    Beruf & Mathe\\
    \hline
  \end{tabular}
  \caption{Mein Lebenslauf}
  \label{tab:meinlebenslauf}
\end{table}
\footnotetext{im S\"uden}

\section{Ein paar Formeln}

\textit{Hier der Versuch, Gleichungen sch\"on auszurichten}. Mit
\verb$\phantom{{-}2<{}}$, wobei das \verb${-}$ erzielt, dass das \emph{minus}
  als unit\"arer Operator auftritt (vgl. ${-}1$ vs. $0-1$); wusste ich auch
  noch nicht and \verb${}<{}$ sicherstellt, dass das $<$ die richtigen
  Abst\"ande erzeugt.
\begin{equation*}
  f(x) =
  \begin{cases}
    \frac{\sin(x)}{x^2+1}\phantom{\sqrt{1}}\quad &\text{f\"ur}\ \phantom{{-}2<{}} x < -2 \\
    e^{x-1} + x^2 \quad &\text{f\"ur}\ {-}2 \leq x \leq 1 \\
    \sqrt{3 + \ln(x)} \quad & \text{f\"ur } \phantom{{-}}1 \leq x
  \end{cases}
\end{equation*}

\section{Gleichungssysteme, Matrizen, Vektoren}
\begin{enumerate}
\item
Schreiben Sie in \LaTeX\ folgende Darstellung eines linearen Gleichungssystems mit den Unbekannten $x_1,\,x_2,\, x_3$ und $x_4$. 
\[
 \begin{array}{rrrrrr}
    4x_1  &        & + 3x_3  &  - x_4 & = & 6\\
          & 2x_2   & - 4x_3  &  + x_4  & = & 0\\ 
    x_1   & + 4x_2 &         &  - x_4  & = & -4\\
          &        & - 3x_3  &  + 8x_4 & = & 1
 \end{array}  
\]
\item Notieren Sie das Gleichungssystem aus (a) in Matrix-Vektor-Schreibweise: $A x = b$ mit Klammern.
  \begin{equation*}
    \begin{bmatrix}
      4  &  0     & \phantom{-}3  &  {-}1\\
      0  & 2   & {-}4  &  \phantom{-}1\\ 
      1   & 4 &     \phantom{{-}}0   &  {-}1 \\
      0   &  0     & {-}3  &  \phantom{-}8
    \end{bmatrix}
    \begin{bmatrix}
      x_1\\x_2\\x_3\\x_4
    \end{bmatrix}
    =
    \begin{bmatrix}
      \phantom{{-}}6\\\phantom{{-}}0\\{-}4\\\phantom{{-}}1
    \end{bmatrix}
  \end{equation*}
\end{enumerate}


\section{Integrale und Summen}

Notieren Sie in \LaTeX\ die summierte Trapezformel zur N"aherung des bestimmten
Integrals einer stetigen Funktion $f: \mathbb R \to \mathbb R$ im Intervall $[a,b]$ mit \\ $h = \frac{\textstyle \rule{0em}{0.8em} b-a}{\textstyle \rule{0em}{0.8em} n}$ und $n+1$ St"utzstellen $t_i= a + i\cdot h, \; i = 0,1,\ldots, n$ auf zwei Arten: 
\begin{eqnarray*}
  \int\limits_a^bf(t) \operatorname{d} t & \approx & h\cdot  \sum_{i = 0}^{n-1}\left (\frac{ f(t_i)  + f(t_{i+1})}{ 2}\right)
\\
 & \approx & \frac{b-a}{2 n}\left ( f(a)  + f(b) + 2 \cdot \sum_{i = 1}^{n-1} f(t_i)\right)
 \end{eqnarray*}

\end{document} 
